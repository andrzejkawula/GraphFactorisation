\documentclass[12pt,a4paper,titlepage]{article}
\usepackage{graphicx}
\usepackage{graphics}
\usepackage{epsfig}
\usepackage{amsmath}
\usepackage{amssymb}
\usepackage{amsthm}
\usepackage{booktabs}
\usepackage{stmaryrd}
\usepackage{url}
\usepackage{longtable}
\usepackage[figuresright]{rotating}
\usepackage[utf8]{inputenc}
\usepackage[T1]{fontenc}
\usepackage[polish]{babel}
\usepackage{geometry}
\usepackage{pslatex}
\usepackage{ulem}
\usepackage{lipsum}
\usepackage{listings}
\usepackage{url}
\usepackage{Here}
\usepackage{color}
\usepackage[ruled,vlined,linesnumbered]{algorithm2e}
\selectlanguage{polish}
\definecolor{szary}{gray}{0.6}
\setlength{\textwidth}{400pt}
\lstset{numbers=left, numberstyle=\tiny, basicstyle=\scriptsize\ttfamily, breaklines=true, captionpos=b, tabsize=2}

\makeindex

\title{Faktoryzacja iloczynu kartezjańskiego grafów }
\date{30.06.2018}
\author{Andrzej Kawula \\ Promotor: dr Monika Pilśniak}

\begin{document}
\maketitle
\tableofcontents
\newpage
\section{Wprowadzenie}
\section{Wstęp}
\section{Algorytm Faktoryzacji}
\subsection{Faktoryzacja z dodatkowymi informacjami}
\subsection{Nadawanie współrzędnych wierzchołkom}
\subsection{Etykietowanie produktu kartezjańskiego}
\subsection{Etykietowanie produktu w czasie liniowym}
\subsection{Sprawdzanie spójnosci kolorowanie właciwego produktu iloczynu kartezjańskiego}
\subsection{Opis algorytm faktoryzacji}
\subsection{Wprowadzenie}
\section{Implementacja Algorytmu}
\subsection{Wprowadzenie}
\subsection{Dane wejsciowe}
\subsection{Opis pakietów i ważniejszych  klas}
\subsection{Opis algorytm faktoryzacji}
\end{document}