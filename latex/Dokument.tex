\documentclass[12pt,a4paper,titlepage]{article}
\usepackage{graphicx}
\usepackage{graphics}
\usepackage{epsfig}
\usepackage{amsmath}
\usepackage{amssymb}
\usepackage{amsthm}
\usepackage{booktabs}
\usepackage{stmaryrd}
\usepackage{url}
\usepackage{longtable}
\usepackage[figuresright]{rotating}
\usepackage[utf8]{inputenc}
\usepackage[T1]{fontenc}
\usepackage[polish]{babel}
\usepackage{geometry}
\usepackage{pslatex}
\usepackage{ulem}
\usepackage{lipsum}
\usepackage{listings}
\usepackage{url}
\usepackage{Here}
\usepackage{color}
\usepackage[ruled,vlined,linesnumbered]{algorithm2e}
\selectlanguage{polish}
\definecolor{szary}{gray}{0.6}
\setlength{\textwidth}{400pt}
\lstset{numbers=left, numberstyle=\tiny, basicstyle=\scriptsize\ttfamily, breaklines=true, captionpos=b, tabsize=2}

\makeindex

\title{Faktoryzacja iloczynu kartezjańskiego grafów }
\date{30.06.2018}
\author{Andrzej Kawula \\ Promotor: dr Monika Pilśniak}

\begin{document}
\maketitle
\tableofcontents
\newpage
\section{Wprowadzenie}
\section{Wstęp}
\subsection{Informacje wstępne}
Produktem kartezjańskim $G_1 \square G_2 $ grafów $G_1 = (V_1 , E_1 ) $ i $ G_2 =(V_2 , E_2 ) $ jest graf $G = (V, E)$, którego zbiorem wierzchołków jest iloczyn kartezjański wierzchołków grafów $G_1$ i $G_2$ $(V=V_1 \times V_2 )$, natomiast wierzchołki $(x_1, y_1)$ oraz $(x_2, y_2)$ są połączone w grafie $G$ jeżeli $x_1 = x_2$ oraz $y_1 y_2 \in E_2 $ lub $x_1 x_2 \in E_1 $ oraz $y_1 = y_2 $.\\
Iloczyn kartezjański grafów jest działaniem łącznym, przemiennym, z dokładnością do izomorfizu, elementem neutralym działania jest graf $K_1$.\\
Z łączności działania możena zapisać $G_1 \square G_2 \square ... \square G_k = G$ gdzie $G$ jest produktem kartezjańskim grafów $G_1, G_2, ... , G_k$, a nastepnie poetykietować wierzchołki grafu G $k$-elementową listą $(v_1, v_2 , ... v_k )$ gdzie $v_i \in V(G_i)$ dla $1 \leqslant i \leqslant k $. Jeżeli $v$ etykietowny jest przez listę $(v_1, v_2 , ... v_k )$, można zdefiniować rzutowanie $p_i : V \rightarrow V_i $ dla $1 \leqslant i \leqslant k $, które dane jest wzorem $p_i (v) = v_i $, gdzie $v_i$ jest i-tym elemetem listy etykietującej wierzchołek $v$. Wierzchołek $v_i $ ten będzie i-tą współrzędną wierzchołka $v$.  \\
Jeżeli w grafie $G$ dany jest wierzchołek $v$ i rozważone zostaną wierzchołki, które różnią się od wierzchołka $v$ tylko na i-tej pozycji, to podgraf indukownay przez te wierzchołki utowrzy graf izomorficzny  z grafem $G_i$. Podgraf ten będzie nazywany i-tą warsttwą $G_i$ przechodzącą przez wierzchołek $v$ a jego oznaczazeneim będzie $G_i ^v$.\\
Niech $v_0$ będzie wyróżnionym wierzchołkiem w grafie $G$. Warstwy przechodzące przez $v_0$ będą nazywane warstwami jednostkowymi. Wierzchołek $v_0$ należy do każdej warstwy jednostkowej, natomiast zbiory $V(G_i ^{v_0}) \setminus \{v_0\}$ są parami rozłączne dla $1 \leqslant i \leqslant k $.
\\
Tutaj będzie przykład \\
\\
\subsection{Kolorowanie produktu iloczynu kartezjańskiego grafów}
Niech dane będą dwa połączone wierzchołki $u$ oraz $v$ w grafie G. Analizująć współrzędne tych wierzchołków, łatwo można stwierdzić że różnią się one dokładne na jednej pozycji. Niech $i$ oznacza tę pozycję. Wtedy krawędź $uv$ należy do $G_i^v$. Krawędzi $uv$  otrzymuje kolor $i$ czyli $c(uv) = i$, gdzie $c$ jest kolorwaniem własciwym produktu kartezjańskiego. Podsumowując: funkcja $c: E(G) \rightarrow {1,2,...,k}$ jest kolorowaniem własciwym produktu iloczynu kartezjańskiego jeżeli $c(uv) = i $ wtedy i tylko wtedy gdy współrzędne wierzchołków $u$ oraz $v$ różnią się na i-tej pozycji.\\
Każda krawędź należy dokładnie do jednej warstwy. Rozważając podgraf grafu $G$ skaładający się z krawędzi koloru $i$ to każda spójna składowa tego podgrafu będzie oddzielną i-tą warstwą grafu $G$.
\subsection{Lemat o kwadracie}
Niech graf $G$ posiada własciwe pokolorowanie produktu kartezjańskiego. Dane sa dwie połączone krawędzie $e$ i $f$ różnych kolorów. Wówczas istieje dokładnie jeden kwadrat bez przekątnych (graf $C_4$) zawierający $e$ oraz $f$.\\
\textit{Dowód:}\\
Na początku przeanalizowany będzie nastepujący fakt. Każdy trójkąt w grafie $G$ jest pomalowany na ten sam kolor. Wynika to z tego, że dwa pierwsze wierzchołki różnią się na pozcyji i-tej, natomiast jeżeli wspólrzędne trzeciego wierzchołka różniły by się na pozycji $j$-tej która jest różna od $i$ w porównaniu z pierwszym wierzchołkiem i jego współrzędnymi, nie było by możliwosci aby drugi i trzeci wierzchołek byłby by ze sobą połączone, ponieważ ich współrzędne różniły by się na dwóch pozycjach.\\
Na podstawie powyższego stwierdzenia można wnioskować, że każdy kwadrat zawierający co najmniej jedną przekątną jest tego samego koloru.\\
Teraz włąsciwy dowodu lematu. Dane są następujące oznaczenia:\\
$(v_0 , v_1, ... ,v_i, ..., v_j,...,v_k )$  współrzędne wspólnego wierzchołka $v$ krawędzi $e$ oraz $f$\\
$(v_0 , v_1, ... ,v'_i, ..., v_j,...,v_k )$ współrzędne wierzchołka $v_e$ który jest drugim końcem krawędzi $e$\\
$(v_0 , v_1, ... ,v_i, ..., v'_j,...,v_k )$ współrzędnymi wierzchołka $v_f$ który jest drugim końcem krawędzi $f$\\
$v'$ wierzchołek o współrzędnych $(v_0 , v_1, ... ,v_i', ..., v'_j,...,v_k )$\\ 
Łatwo stwierdzić, że wierzchołek $v'$ jest połączony z wierzchołkiem $v_e$ ponieważ ich współrzędne różnią się tylko na pozycji $j$ oraz w grafie $G_j$ istnieje krawędź $v_j v'_j$ ponieważ w grafie $G$ isnieje krawęź $f$. Analogicznie stwierdzia się istnienie krawędzi $v'v_f$, co w połączeniu z faktem, że krawędzie $e$ oraz $f$ są różnego koloru i rozważaniom na temat kwadratów z przekątnymi daje nam tezę lematu. Co więcej na podstawie powyższego rozumowania, można wnioskowaćy że przeciwległe krawędzie w kwadracie mają ten sam kolor, niezależnie czy kwadrat posiada przekątne czy też nie.
\subsection{Lemat o izomorfizmie}
Niech $G=(V, E)$ będzie spójnym grafem, natomiast $E_1 , E_2 , ... , E_k$ podziałem zbioru krawędzi. Niech każda spójna składowa $(V, \cup_{j \neq i}E_j)$ ma dokłdnie jeden punkt współny z każdą spóją skłądową $(V, E_i)$ oraz krawędzie między dwoma składowymi $(V, E_i)$ wyznaczają izomorfizm między tymi składowymi (jeżeli takie krawędzie istnieją). Wtedy:\\
$G=\Pi G_i $ \\
gdzie $G_i $ jest dowolną, spójną składową $(V, E_i)$.
 
\subsection{Lemat o udoskonaleniu faktoryzacji iloczynu kartezjańskiego}

\section{Algorytm Faktoryzacji}
\subsection{Faktoryzacja z dodatkowymi informacjami}
\subsection{Nadawanie współrzędnych wierzchołkom}
\subsection{Etykietowanie produktu kartezjańskiego}
\subsection{Etykietowanie produktu w czasie liniowym}
\subsection{Sprawdzanie spójnosci kolorowanie właciwego produktu iloczynu kartezjańskiego}
\subsection{Opis algorytm faktoryzacji}
\subsection{Wprowadzenie}
\section{Implementacja Algorytmu}
\subsection{Wprowadzenie}
\subsection{Dane wejsciowe}
\subsection{Opis pakietów i ważniejszych  klas}
\subsection{Opis algorytm faktoryzacji}
\end{document}